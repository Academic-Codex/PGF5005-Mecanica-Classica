\documentclass[a4paper]{article}
\usepackage{student}
\usepackage{graphicx}
\usepackage{caption}
\usepackage[version=4]{mhchem}
\usepackage{tikz}
\usetikzlibrary{shapes.geometric, arrows.meta, positioning, decorations.pathreplacing}
\usepackage{enumitem}
\usepackage[utf8]{inputenc}
\usepackage{amsmath}
\pagestyle{plain}

\tikzstyle{arrow} = [thick,->,>=stealth]

% Definindo o estilo de destaque com linhas pontilhadas
\tikzstyle{highlight} = [draw, dashed, thick, rectangle, rounded corners, inner sep=0.2cm, orange]


\tikzstyle{startstop} = [
    rectangle, rounded corners, minimum width=0.5cm,
    text centered, draw=black, fill=blue!10, font=\small
]
\tikzstyle{startstop_S} = [
    rectangle, rounded corners, minimum width=0.5cm, minimum height=0.8cm,
    text centered, draw=black, fill=green!30, font=\small
]
\tikzstyle{decision} = [
    diamond, aspect=2, draw=black, fill=orange!15, align=center,
    text centered, inner sep=0pt, font=\small
]
\tikzstyle{decision_S} = [
    diamond, aspect=2, draw=black, fill=orange!30, align=center,
    text centered, inner sep=0pt, font=\small
]
\tikzstyle{arrow} = [thick,->,>=stealth]



% Metadata
\date{\today}
\setmodule{PGF5005/IFUSP: Mecânica Clássica I. \\ Prof.: Matheus Jean Lazarotto} 
\setterm{2o. semestre, 2025}

%-------------------------------%
% Other details
% TODO: Fill these
%-------------------------------%
\title{Estudo Dirigido 01 - ??/09}
\setmembername{Nara Avila Moraes}  % Fill group member names
\setmemberuid{5716734}  % Fill group member uids (same order)

%-------------------------------%
% Add / Delete commands and packages
% TODO: Add / Delete here as you need
%-------------------------------%
\usepackage{amsmath,amssymb,bm}

\newcommand{\KL}{\mathrm{KL}}
\newcommand{\R}{\mathbb{R}}
\newcommand{\E}{\mathbb{E}}
\newcommand{\T}{\top}

\newcommand{\expdist}[2]{%
        \normalfont{\textsc{Exp}}(#1, #2)%
    }
\newcommand{\expparam}{\bm \lambda}
\newcommand{\Expparam}{\bm \Lambda}
\newcommand{\natparam}{\bm \eta}
\newcommand{\Natparam}{\bm H}
\newcommand{\sufstat}{\bm u}

% Main document
\begin{document}
    % Add header
    \header{}

\textbf{Questão 04 - Pêndulo Simples}
\begin{center}
\begin{itemize}
    \item[(4.2)] Realize a implementação computacional do método de Euler convencional para resolver as equações de movimento do pêndulo simples. Anexe o código do programa desenvolvido ao seu trabalho.
    \item[(4.3)]        
    \item[(4.4)]
    \item[(4.5)]
    \item[(4.7)]    
    \item[(4.8)]
    \item[(4.9)]
    \item[(4.10)]
\end{itemize}
\end{center}

\textbf{Questão 05 - Hamiltoniana de Henon-Heiles}
\begin{center}
\begin{itemize}
    \item[(5.2)]
    \item[(5.3)]
    \item[(5.4)]
    \item[(5.5)]
    \item[(5.6)]
    \item[(5.7)]
\end{itemize}
\end{center}

    \begin{answer}[Ítem 4.2]
    \end{answer}

    \begin{answer}[Ítem 4.3]
    \end{answer}

    \begin{answer}[Ítem 4.4]
    \end{answer}

    \begin{answer}[Ítem 4.5]
    \end{answer}

    \begin{answer}[Ítem 4.7]
    \end{answer}

    \begin{answer}[Ítem 4.8]
    \end{answer}

    \begin{answer}[Ítem 4.9]
    \end{answer}

    \begin{answer}[Ítem 4.10]
    \end{answer}

    \begin{answer}[Ítem 5.2]
    \end{answer}

    \begin{answer}[Ítem 5.3]
    \end{answer}

    \begin{answer}[Ítem 5.4]
    \end{answer}

    \begin{answer}[Ítem 5.5]
    \end{answer}

    \begin{answer}[Ítem 5.6]
    \end{answer}

    \begin{answer}[Ítem 5.7]
    \end{answer}
\end{document}
